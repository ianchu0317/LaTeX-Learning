\documentclass[12pt, a4paper]{article}
\usepackage{fullpage} % Ajustar margen del archivo a 1in
\usepackage{graphicx} % Required for inserting images

% Título del documento
\title{Resumen indeterminaciones límites}
\author{Ian Chen}
\date{May 2024}


% Inicio documento
\begin{document}
% Título
\maketitle
\begin{abstract}
Resumen de las distintas indeterminaciones que existen y surgen durante  el análisis de funciones cuando se utiliza límites.
\end{abstract}
\vspace{1cm}
\tableofcontents
\newpage

%Introducción del documento
\section{Introduction}
\subsection{¿Qué es una indeterminación?}
En el cálculo, una indeterminación ocurre cuando intentamos evaluar un límite y obtenemos una forma que no nos permite determinar directamente el valor del límite. Estas formas indeterminadas se presentan comúnmente como resultados de ciertas operaciones algebraicas y funciones que, sin un análisis más detallado, no nos dan información suficiente para determinar el límite.

\subsection{Tipos de indeterminaciones}
\begin{enumerate}
    \item \textbf{$\displaystyle \frac{0}{0}$}: Ocurre cuando tanto el numerador como el denominador tienden a cero.
    \item $\displaystyle \frac{\infty}{\infty}$ Ocurre cuando tanto el numerador como el denominador tienden a infinito.
    \item $\displaystyle 0 \cdot \infty$: Ocurre cuando un factor tiende a cero y el otro tiende a infinito.
    \item $\displaystyle \infty - \infty$: Ocurre cuando dos cantidades que tienden a infinito se restan.
    \item $\displaystyle 0^0$: Ocurre cuando una base que tiende a cero se eleva a una potencia que también tiende a cero.
    \item $\displaystyle \infty^0$: Ocurre cuando una base que tiende a infinito se eleva a una potencia que tiende a cero.
    \item $\displaystyle 1^\infty$: Ocurre cuando una base que tiende a uno se eleva a una potencia que tiende a infinito.
\end{enumerate}
Para resolver estas indeterminaciones, a menudo utilizamos técnicas adicionales, tales como:

\begin{itemize}
    \item \textbf{Factorización y simplificación}: Simplificar las expresiones algebraicas para eliminar la indeterminación.
    \item \textbf{Racionalización}: Multiplicar por una forma conjugada para simplificar la expresión.
    \item \textbf{Regla de L'Hôpital}: Usar derivadas para evaluar límites que tienen la forma indeterminada $\displaystyle \frac{0}{0} $ o $\displaystyle \frac{\infty}{\infty}$.
    \item \textbf{Series de Taylor}: Expansiones en series para aproximar funciones y resolver indeterminaciones.
    \item \textbf{Cambios de variable}: Introducir nuevas variables para reescribir el límite de una manera más manejable.
\end{itemize}

(ESTE PARRAFO NO ES CHATGPT)
Las transformaciones de las funciones sólo es aplicable para el estudio de límites y no su evaluación. O sea, en funciones reales (cuando queremos operar $f(2)=\frac{0}{0}$ por ejemplo) no es posible operar con transformaciones porque se dan valores exactos. Durante el estudio del límite se estudia una tendencia (ex. $\lim_{x\to2}f(x)=\frac{0}{0}$), por lo cual se estudian todos los valores cercanos al número estudiado pero nunca el número en sí. Por eso se puede trabajar con transformaciones de las funciones dentro de ese ámbito. 

% Indeterminacion 1
\section{Indeterminación $\displaystyle \frac{0}{0}$}
Para este tipo de indeterminación se puede utilizar la factorización o la racionalización para despejar la indeterminación.\\[6pt]
\textbf{Ejemplos}:

\vspace{0.5cm}
\begin{enumerate}
% factorización 0/0
\item Método de factorización
\begin{eqnarray}
&\lim_{x\to3}&\left(\frac{x^2-2x-3}{4x-12}\right)\\
&\lim_{x\to3}&\left(\frac{(x-3)(x+1)}{4\left(x-3\right)}\right)\\
&\lim_{x\to3}&\left(\frac{x+1}{4}\right) = 1
\end{eqnarray}
Se factoriza tanto numerador como denominador y simplificar términos en común para llegar a una expresión más fácil de operar.

\setcounter{equation}{0}
\vspace{1cm}

% racionalizacion 0/0
\item Método de racionalización
\begin{eqnarray}
    &\lim_{x\to0}&\left(\frac{\sqrt{1-x}-\sqrt{1+x}}{x}\right)\\
    &\lim_{x\to0}&\left(\frac{\sqrt{1-x}-\sqrt{1+x}}{x} \cdot \frac{\sqrt{1-x}+\sqrt{1+x}}{\sqrt{1-x}+\sqrt{1+x}}\right)\\
    &\lim_{x\to0}&\left(\frac{(1-x)-(1+x)}{x\left(\sqrt{1-x}+\sqrt{1+x}\right)}\right)\\
    &\lim_{x\to0}&\left(\frac{-2x}{x\left(\sqrt{1-x}+\sqrt{1+x}\right)}\right)\\
    &\lim_{x\to0}&\left(\frac{-2}{\sqrt{1-x}+\sqrt{1+x}}\right) = -1
\end{eqnarray}
Parte del problema son las raíces del dividendo, por lo cual se multiplica y divide por el conjugado para hallar mejor expresión y salvar el límite.

\setcounter{equation}{0}

%L'Hopital
\item L'Hopital\\[6pt]
El límite del cociente entre la derivada del numerador y la derivada del denominador es equivalente al límite del cociente entre el numerador y el denominador. $\displaystyle \lim_{x\to\alpha}\frac{f(x)}{g(x)}=\lim_{x\to\alpha}\frac{f'(x)}{g'(x)}$
\begin{eqnarray}
&\lim_{x\to3}&\left(\frac{x^2-2x-3}{4x-12}\right)\\
&\lim_{x\to3}&\left(\frac{\left(x^2-2x-3\right)'}{\left(4x-12\right)'}\right)\\
&\lim_{x\to3}&\left(\frac{2x-2}{4}\right) = 1
\end{eqnarray}
\end{enumerate}

% Indeterminación 2
\section{Indeterminación $\displaystyle \frac{\infty}{\infty}$}
Para este tipo de indeterminación se puede dividir por polinomio del mayor grado presente, o simplemente factorizar ambos numerador y el denominador.\\[6pt]
Ejemplos:

\vspace{0.5cm}
\begin{enumerate}
    \item Factorización
    \begin{eqnarray}
        \lim_{x\to\infty}\left(\frac{x^2-1}{x-2}\right)\\
        \lim_{x\to\infty}\left(\frac{x\left(x-\frac{1}{x}\right)}{x\left(1-\frac{2}{x}\right)}\right)\\
        \lim_{x\to\infty}\left(\frac{x-\frac{1}{x}}{1-\frac{2}{x}}\right) = \infty
    \end{eqnarray}
    
    \item L'Hopital
    \begin{eqnarray}
        \lim_{x\to\infty} \left(\frac{x^2-1}{x-2}\right)\\
        \lim_{x\to\infty} \left(\frac{\left(x^2-1\right)'}{\left(x-2\right)'}\right)\\
        \lim_{x\to\infty}\left(\frac{2x}{1}\right) = \infty
    \end{eqnarray}
\end{enumerate}
\newpage

\section{Indeterminación $\displaystyle 0 \cdot \infty$}
La mejor forma para evitar la indeterminación $\displaystyle 0 \cdot \infty$ es transformarla en la indeterminación $\displaystyle \frac{\infty}{\infty}$ o $\displaystyle \frac{0}{0}$. Si lo hacemos, ya sabemos cómo proceder. Además, podemos aplicar L'Hôpital en estas indeterminaciones.\\[6pt]
Supongamos que $f(x)$ tiende a $0$ y $g(x)$ tiende a $\infty$. Entonces, la indeterminación $0 \cdot \infty$ aparece en el producto $f(x) \cdot g(x)$: $$\lim_{x\to\alpha}f(x) \cdot g(x)$$ Como límite de $f(x)$ es $0$, entonces su inverso multiplicativo será $\infty$:
\setcounter{equation}{0}
\begin{eqnarray}
    \lim_{x\to\alpha}f(x) = 0\\
    \lim_{x\to\alpha}\frac{1}{f(x)} = \infty
\end{eqnarray}
Por lo tanto sólo hace falta reescribir el producto como un cociente: $$\lim_{x\to\alpha}\frac{g(x)}{\frac{1}{f(x)}} = \frac{\infty}{\infty}$$
Analógicamente para $g(x)$: $$\lim_{x\to\alpha}\frac{f(x)}{\frac{1}{g(x)}} = \frac{0}{0}$$

% Indeterminación 4
\section{Indeterminación $\displaystyle \infty - \infty$}
Para este tipo de indeterminaciones se puede comparar grados de infinitos en los polinomios y ver si tiende a $+\infty$ o $-\infty$, o en caso de presentar raíces, multiplicar y dividir por el conjugado.

%indeterminación 
\section{Indeterminación $\displaystyle 0^0$}
Utilización de propiedades de logaritmo.

\section{Indeterminación $\displaystyle \infty^0$}
Utilización de propiedades de logaritmo.

\section{Indeterminación $\displaystyle 1^\infty$} 
Transformar la expresión hasta llegar a la identidad $\displaystyle \lim_{algo\to\infty}\left(1+\frac{1}{algo}\right)^{algo}$


\end{document}
