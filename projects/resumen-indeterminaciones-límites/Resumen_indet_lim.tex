\documentclass[12pt, a4paper]{article}
\usepackage{fullpage} % Ajustar margen del archivo a 1in
\usepackage{graphicx} % Required for inserting images

% Título del documento
\title{Resumen indeterminaciones límites}
\author{Ian Chen}
\date{May 2024}


% Inicio documento
\begin{document}
% Título
\maketitle
\begin{abstract}
Resumen de las distintas indeterminaciones que existen y surgen durante  el análisis de funciones cuando se utiliza límites.
\end{abstract}
\vspace{1cm}
\tableofcontents
\newpage

%Introducción del documento
\section{Introduction}
\subsection{¿Qué es una indeterminación?}
En el cálculo, una indeterminación ocurre cuando intentamos evaluar un límite y obtenemos una forma que no nos permite determinar directamente el valor del límite. Estas formas indeterminadas se presentan comúnmente como resultados de ciertas operaciones algebraicas y funciones que, sin un análisis más detallado, no nos dan información suficiente para determinar el límite.

\subsection{Tipos de indeterminaciones}
\begin{enumerate}
    \item \textbf{$\displaystyle \frac{0}{0}$}: Ocurre cuando tanto el numerador como el denominador tienden a cero.
    \item $\displaystyle \frac{\infty}{\infty}$ Ocurre cuando tanto el numerador como el denominador tienden a infinito.
    \item $\displaystyle 0 \cdot \infty$: Ocurre cuando un factor tiende a cero y el otro tiende a infinito.
    \item $\displaystyle \infty - \infty$: Ocurre cuando dos cantidades que tienden a infinito se restan.
    \item $\displaystyle 0^0$: Ocurre cuando una base que tiende a cero se eleva a una potencia que también tiende a cero.
    \item $\displaystyle \infty^0$: Ocurre cuando una base que tiende a infinito se eleva a una potencia que tiende a cero.
    \item $\displaystyle 1^\infty$: Ocurre cuando una base que tiende a uno se eleva a una potencia que tiende a infinito.
\end{enumerate}
Para resolver estas indeterminaciones, a menudo utilizamos técnicas adicionales, tales como:

\begin{itemize}
    \item \textbf{Factorización y simplificación}: Simplificar las expresiones algebraicas para eliminar la indeterminación.
    \item \textbf{Racionalización}: Multiplicar por una forma conjugada para simplificar la expresión.
    \item \textbf{Regla de L'Hôpital}: Usar derivadas para evaluar límites que tienen la forma indeterminada $\displaystyle \frac{0}{0} $ o $\displaystyle \frac{\infty}{\infty}$.
    \item \textbf{Series de Taylor}: Expansiones en series para aproximar funciones y resolver indeterminaciones.
    \item \textbf{Cambios de variable}: Introducir nuevas variables para reescribir el límite de una manera más manejable.
\end{itemize}

% Indeterminacion 1
\section{Indeterminación $\displaystyle \frac{0}{0}$}
Para este tipo de indeterminación se puede utilizar la factorización o la racionalización para despejar la indeterminación.\\[6pt]
Ejemplos:

\vspace{0.5cm}
a) Método de factorización

b) Método de racionalización

c) L'Hopital


% Indeterminación 2
\section{Indeterminación $\displaystyle \frac{\infty}{\infty}$}
Para este tipo de indeterminación se puede dividir por polinomio del mayor grado presente, o simplemente factorizar ambos numerador y el denominador.\\[6pt]
Ejemplos:

\vspace{0.5cm}
a) Factorización

b) L'Hopital

\section{Indeterminación $\displaystyle 0 \cdot \infty$}


\section{Indeterminación $\displaystyle \infty - \infty$}
\section{Indeterminación $\displaystyle 0^0$}
\section{Indetermianacióm $\displaystyle \infty^0$}
\section{Indeterminación $\displaystyle 1^\infty$} 



\end{document}
