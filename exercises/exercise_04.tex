\documentclass{article}
\usepackage{graphicx} % Required for inserting images
\usepackage{amsmath} % for align
\usepackage{hyperref}
\hypersetup{colorlinks=true, linkcolor=blue}

% Title setting
\title{Hello World!}
\author{Ian Chen}
\date{June 1, 2024}

\begin{document}
\maketitle

%%%%%%% FIRST SECTION %%%%%%%
\section{Getting Started}
\textbf{Hello World}! Today I am learning \LaTeX. \LaTeX is a great program for writing math. I can write in line math such as $a^2+b^2=c^2$. I can also give equations their own space:
\begin{align}
    \gamma^2+\theta^2=\omega^2
\end{align}
''Maxwell's equations" are named for James Clark Maxwell and are as follow:

\begin{align}
\Vec{\Delta} \cdot \Vec{E} &= \,\,\frac{\phi}{\epsilon_0}  &\text{Gauss's Law} \label{gauss1}\\
\Vec{\Delta} \cdot \Vec{B} &= \,\,0 &\text{Gauss's Law for Magnetism} \label{gauss2}\\
\Vec{\Delta} \times \Vec{E} &= \,\,-\frac{\delta \Vec{B}}{\delta t} &\text{Faraday's Law of Induction} \label{faraday1}\\
\Vec{\Delta} \times \Vec{B} &= \,\,\mu_0 \left(\epsilon_0 \frac{\delta \Vec{E}}{\delta t} + \Vec{J}\right) &\text{Ampere's Circuital Law} \label{ampere1}
\end{align}

Equations \ref{gauss1}, \ref{gauss2}, \ref{faraday1} and \ref{ampere1} are some of the most important in Physics.

%%%%%% Second Section %%%%%%
\section{What about Matrix Equations?}



\end{document}
